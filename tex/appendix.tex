% !TEX root = ../thesis-sample.tex
\appendix
\chapter{Methods}
Here is how to implement the methods.

\section{Bisection}
The easiest method.

\begin{equation}\label{eq:sum}
    x_k = \frac{a_k+b_k}{2}
\end{equation}

\section{False Position}
The next one.

\chapter{Using Appendices}    \label{app:appendix}

This section might be referencing code and options that no longer exist in this version of the thesis class.
It should also be updated as well.

This appendix contains the portion of the users' manual that describes
how to use appendices with this template.  It is put in this appendix
rather than in Chapter~\cref{chap:intro} simply so that there are two
appendices, so that a list of appendices can appear earlier in the
document.

\section{Starting the Appendices}
Actually, using appendices is quite simple.  Immediately after the end
of the last chapter and before the start of the first appendix, simply
enter the command \verb|\appendix|.  This will tell \LaTeX~to change
how it interprets the commands \verb|\chapter|, \verb|\section|,
\textit{etc.}

Each appendix is actually a chapter, so once the \verb|\appendix|
command has been called, start a new appendix by simply using the
\verb|\chapter| command.

Note that the \verb|\appendix| command should be called only
once--not before the start of each appendix.

