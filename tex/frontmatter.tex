% !TEX root = ../thesis-sample.tex

% --------- FRONT MATTER PAGES ---------------------
% Title of the thesis
\title{Motivating the inclusion of construct validity in social media measurement: Translating trust in vaccines from surveys to Twitter}

% Author name
\author{Michael C. Smith}

% Previous degrees
\bsdepartment{Computer Science}
\bsschool{The Johns Hopkins University} %made change in .cls file from B.S. to B.Eng.
\bsgrad{May 2011}

\msdepartment{Computer Science}
\msschool{The Johns Hopkins University}
\msgrad{December 2014}
\showmsdegree % you can show or hide the MS degree line 
%\hidemsdegree

% PhD degree commands
% Committee
\showcommitteepage % hide this page if you're doing a MS thesis
%\hidecommitteepage 
\committee{ %
David A. Broniatowski, Associate Professor of Engineering Management and Systems Engineering, The George Washington University, Dissertation Director\\ % remember to add a space between committee members

Thomas A. Mazzuchi, Professor of Engineering Management and Systems Engineering, George Washington University, Committee Member \\

Dr. Sandra C. Quinn, Professor of Family Science,  University of Maryland, 
Committee Member \\

J. Rene van Dorp, Professor of Engineering Management and Systems Engineering, George Washington University, Committee Member \\

Mark Dredze, John C. Malone Associate Professor in the Department of Computer Science, The Johns Hopkins University, External Examiner \\
}

% Chair must be entered separately for formatting reasons.
\chair{David A. Broniatowski}
\chairtitle{Associate Professor of Engineering Management and Systems Engineering}
% Department
\department{Engineering Management and Systems Engineering}

% \phdgrad{TBD}
\defensedate{2019-12-18}
% Year of completion for copyright page and perhaps other places
\year=2020

% Copyright page
%\copyrightholder{Someone else}

% Dedication
\dedication{ %
%Include a fancy quote or dedication
\begin{center}
    \emph{I hope I make you proud.}
\end{center}
}

% Acknowledgments
\acknowledgments{
    I would like to acknowledge all of the people who helped me get to this point, because this document is the capstone of the last 5 years of my career.\\
    
}

% -----------------------------------------------------------------
% Typically only one of Preface/Foreword/Prologue would be in your thesis.
% To choose one simply delete the others and they will automatically disappear

% Preface
\preface{
    This is the preface. 
    It's another front matter page that offers additional detail into your work.
    Typically, only one (preface OR prologue OR foreword) is used. 
    You can remove the other sections by deleting them inside \texttt{tex/frontmatter.tex} or using the appropriate show or hide commands.
}

\prologue{
    This is the prologue. 
    It's another front matter page that offers additional detail into your work.
    Typically, only one (preface OR prologue OR foreword) is used. 
    You can remove the other sections by deleting them inside \texttt{tex/frontmatter.tex} or using the appropriate show or hide commands.
}

\foreword[2]{
    This is the foreword. 
    It's another front matter page that offers additional detail into your work.
    Typically, only one (preface OR prologue OR foreword) is used. 
    You can remove the other sections by deleting them inside \texttt{tex/frontmatter.tex} or using the appropriate show or hide commands.
}
% ----------------------------------------------------------------------

% commands to show or hide front matter pages

\showcopyright
\showabstract
\showcommitteepage
\showdedication
\showacknowledgments
\showpreface
\hideprologue
\hideforeword

% ------------ TABLE OF CONTENTS ----------------------
% add 'chapter' preceding the snum
\renewcommand{\cftchappresnum}{\chaptername\space}
\renewcommand{\cftchapaftersnum}{:}
\setlength{\cftchapnumwidth}{\widthof{Appendix AAA~ }}
% change this to 'appendix' by hacking the \appendix command
\makeatletter
\g@addto@macro\appendix{%
  \addtocontents{toc}{%
    \protect\renewcommand{\protect\cftchappresnum}{\appendixname\space}%
  }%
}
\makeatother

% Commands to hide or show lists of figures, tables, etc.
\showlistoffigures
\showlistoftables
\hidenomenclature

% --------- ACRONYMS and SYMBOLS ------------------------------
% TODO Deprecate the entire acronym package and switch to glossaries

% You can either use the acronym or glossaries package (both work)
% Definition of any abbreviations used.
\abbreviations{
    \acro{CRTBP}{Circular Restricted Three Body Problem}
    \acro{NSA}{National Security Agency}
    \acro{SSME}{Space Shuttle Main Engine}
}
% call an abbreviation using \ac{abbrev}

% symbols and acronyms only show up when used in the text
\symbols{
    \acro{J}{Moment of Inertia}
}       

% if you want acronym (simpler) then change these to show
\hidelistofabbreviations
\hidelistofsymbols

% if you want glossaries (more powerful) then leave above as hide
% GLOSSARIES package options - automatically turns off front pages from acronym package

% acronymns and symbols are basically the same, but there are two provided 
% locations where they can show up
\setabbreviationstyle[acronym]{long-short}
\setabbreviationstyle[abbreviation]{long-short}
\makeglossaries
% you can hide/show the glossaries page
\showglossarieslistofabbreviations
\showglossarieslistofsymbols
\showglossariesglossaryofterms

% acronyms defined in glossaries
\newabbreviation{crtbp}{CRTBP}{Circular Restricted Three Body Problem}
\newabbreviation{lidar}{LIDAR}{Light Detection and Ranging}
% defining abbreviations like this allows for autocompletion
\newglossaryentry{filo}{
    name={FILO},
    type=\glsxtrabbrvtype,
    description={first in last out},
    first={first in last out (FILO)}
}

% glossary entries
\newglossaryentry{linux}{
    name=Linux,
    description={is a generic term referring to the family of Unix-like computer operating systems that use the Linux kernel},
    plural=Linuces
}

\newglossaryentry{matrix}{
    name={matrix},
    plural={matrices},
    description={rectangular array of quanttities}
}

% symbols
\newglossaryentry{M}{
    type=symbols,
    name={\ensuremath{M}},
    sort=M,
    description={a \gls{matrix}}
}

\newglossaryentry{F}{
    type=symbols,
    name={\ensuremath{F}},
    sort=F,
    description={External Force}
}
% Some abstract text
\abstract{
This is the abstract. 
It contains some random text from the \texttt{lipsum} package. 
You may safely remove the \texttt{lipsum} package once you write your thesis.

\lipsum[1]
}
